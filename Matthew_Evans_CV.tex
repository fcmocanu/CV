\documentclass[11pt,a4paper,sans]{moderncv}        % possible options include font size ('10pt', '11pt' and '12pt'), paper size ('a4paper', 'letterpaper', 'a5paper', 'legalpaper', 'executivepaper' and 'landscape') and font family ('sans' and 'roman')

%\usepackage{libertine}
%\usepackage[scaled=0.9]{inconsolata}
\usepackage{fontspec}
\setsansfont[BoldFont={Archivo Narrow Bold}]{Archivo Narrow}
\setmonofont[Scale=0.9, BoldFont={Iosevka Term Medium}]{Iosevka Term Medium}
% moderncv themes
\moderncvstyle{classic}                             % style options are 'casual' (default), 'classic', 'banking', 'oldstyle' and 'fancy'
\moderncvcolor{blue}                               % color options 'black', 'blue' (default), 'burgundy', 'green', 'grey', 'orange', 'purple' and 'red'
\definecolor{color2}{RGB}{12, 69, 129}
\usepackage[unicode,pdfencoding=auto]{hyperref}
%\definecolor{color2}{RGB}{56,115,178}
%\usepackage{doi}
\usepackage[style=nature,sorting=ydnt]{biblatex}
%\bibliography{pubs}
\addbibresource{pubs.bib}

%hyphenation
\tolerance=1
\emergencystretch=\maxdimen
\hyphenpenalty=10000
\hbadness=10000

% Reverse numbering in publications list
\newcounter{entrycount}
\AtDataInput{\stepcounter{entrycount}}
\DeclareFieldFormat{labelnumber}{\mkrevbibnum{#1}}
\newcommand{\mkrevbibnum}[1]{\number\numexpr\value{entrycount}+1-#1}

\newcommand\colourhref[3][color2]{\href{#2}{\color{#1}#3}}
\newcommand{\doi}[1]{DOI: {\colourhref{https://dx.doi.org/#1}{\texttt{#1}}}}
%\renewcommand{\familydefault}{\sfdefault}         % to set the default font; use '\sfdefault' for the default sans serif font, '\rmdefault' for the default roman one, or any tex font name
\nopagenumbers{}                                  % uncomment to suppress automatic page numbering for CVs longer than one page

\cfoot{\emph{Last modified: \today}}

\usepackage{etoolbox}
\makeatletter
\patchcmd{\makecvtitle}% <cmd>
  {\httplink{\@homepage}}% <search>
  {{\ifx\@homepage@shorthand\relax
     \httplink{\@homepage}% Used \homepage{<URL>}
   \else
     \httplink[\@homepage@shorthand]{\@homepage}% Used \homepage[<desc>]{<URL>}
   \fi}}% <replace>
  {}{}% <succes><failure>
\patchcmd{\thebibliography}
{\advance\leftmargin\labelsep}
  {\labelsep=0.7cm \advance\leftmargin\labelsep}{}{}
\RenewDocumentCommand{\homepage}{o m}{%
  \let\@homepage@shorthand\relax%
  \providecommand\@homepage{#2}%
  \IfNoValueF{#1}{\def\@homepage@shorthand{#1}}%
}
\makeatother

% adjust the page margins
\usepackage[scale=0.8]{geometry}
%\setlength{\hintscolumnwidth}{3cm}                % if you want to change the width of the column with the dates
%\setlength{\makecvheadnamewidth}{12cm}            % for the 'classic' style, if you want to force the width allocated to your name and avoid line breaks. be careful though, the length is normally calculated to avoid any overlap with your personal info; use this at your own typographical risks...

% personal data
\name{Matthew}{Evans}
%\title{Resumé title}                               % optional, remove / comment the line if not wanted
\email{me388@cam.ac.uk}                               % optional, remove / comment the line if not wanted
\homepage[www.tcm.phy.cam.ac.uk/\textasciitilde me388]{www.tcm.phy.cam.ac.uk/~me388/}
\social[github]{ml-evs}                           % optional, remove / comment the line if not wanted
\quote{\texttt{ab initio calculations \textbullet\, energy storage\\ crystal structure databases \textbullet\, software development}}

\renewcommand*{\bibliographyitemlabel}{[\arabic{enumiv}]}

\begin{document}
%-----       resume       ---------------------------------------------------------
\pagestyle{empty}
\makecvtitle
\section{\texttt{EDUCATION}}
\cventry{2016--}{PhD Physics}{Theory of Condensed Matter Group}{University of Cambridge}{\textit{Expected graduation 2019}}{}  % arguments 3 to 6 can be left empty
\cventry{2015--2016}{MPhil Scientific Computing}{}{University of Cambridge}{\textit{Distinction}}{}  % arguments 3 to 6 can be left empty
\cventry{2011--2015}{MPhys Physics with Theoretical Physics}{}{University of Manchester}{\textit{First Class (Hons)}}{}

\section{\texttt{(RESEARCH INTERESTS + EXPERIENCE)}}
\cvitem{PhD}{\textbf{Crystal structure prediction for next-generation energy storage applications}}
\cvitem{}{with Dr Andrew Morris \emph{(University of Cambridge)}}
\cvitem{}{Discovery and computational characterisation of novel high-capacity anode materials for Li-, Na- and K-ion batteries, using \emph{ab initio} random structure searching (AIRSS) and evolutionary approaches, implemented in the \colourhref{http://www.tcm.phy.cam.ac.uk/~me388/ilustrado}{\texttt{ilustrado}} package. PhD research undertaken as a member of the EPSRC CDT for Computational Methods in Materials Science.}

\cvitem{MPhil}{\textbf{High-throughput \emph{ab initio} materials discovery}}
\cvitem{}{with Dr Andrew Morris \emph{(University of Cambridge)}}
\cvitem{}{Database approaches to materials design; wrote a software package, \colourhref{http://www.tcm.phy.cam.ac.uk/~me388/matador}{\texttt{matador}}, to aggregate and {analyse} the results of first-principles calculations.}

\cvitem{MPhys}{\textbf{Electronic structure of defects in graphene superlattices}}
\cvitem{}{with Prof Francisco Guinea \emph{(University of Manchester)}}
\cvitem{}{Nearly-free electron model of graphene/h-BN superlattices with arbitrary defects included via Green's function  methods. Awarded Tessella Prize for software development.}

\cvitem{UG}{\textbf{Interactions of quantised vortices in superfluid helium}}
\cvitem{}{with Dr Paul Walmsley \& Prof Andrei Golov \emph{(University of Manchester)}}
\cvitem{}{Spent two summers developing \colourhref{https://github.com/ml-evs/vfmcpp}{\texttt{vfmcpp}}, a C++/OpenMP implementation of the vortex filament model of superfluid helium, to study microscopic vortex dynamics and reconnection events \cite{PhysRevFluids.1.044502}. }

\cvitem{UG}{\textbf{Hard sphere packing of nanotube-encapsulated fullerenes}}
\cvitem{}{with Dr Ho-Kei Chan \& Prof Elena Besley \emph{(University of Nottingham)}}
\cvitem{}{Application of a novel hard sphere packing regime to study CNT-encapsulated C$_{60}$ molecules.}

\section{\texttt{TEACHING}}
\cvitem{2016--}{Supervisor: Part IB Electromagnetism, Dynamics and Thermodynamics, \emph{(Selwyn College)}.}
\cvitem{2016--}{Demonstrator: Part IB Computational Physics (C++), \emph{(Cavendish Laboratory)}.}
\cvitem{2016--2017}{Volunteer: Key Stage 2 Code Club \emph{(Ridgefield Primary School, Cambridge)}.}
\cvitem{2016}{Demonstrator: Graduate-level Electronic Structure, \emph{(Cavendish Laboratory)}.}
\cvitem{2016}{Demonstrator: CASTEP Workshop \emph{(Oxford)} (x2), HPC Autumn Academy \emph{(Cambridge)}.}
\cvitem{2012--2015}{Tutor: GCSE Maths \& Key Stage 2 Programming for \colourhref{http://thetutortrust.org/}{The Tutor Trust}, \emph{(Manchester)}.
\begin{itemize}
    \item[--] Provided tuition to small groups and `looked after children' across 15 schools.
\item[--] Helped lead a successful pilot to teach primary school children programming using Scratch.\end{itemize}
}

\section{\texttt{COMPUTING}}
\centering Exposure: \textbf{Daily}, Intermittant, \textit{Occasional}.\\[0.2em]
\cvdoubleitem{Languages}{\textbf{Python}, Fortran, C++, \textit{Rust}}{Databases}{\textbf{MongoDB}, \textit{SQL}}
\cvdoubleitem{DFT}{\textbf{CASTEP}, Quantum Espresso}{Packages}{\textbf{NumPy, spglib, scikit-learn}}
\cvdoubleitem{Platforms}{\textbf{Linux}, \textit{*nix}}{HPC Facilities}{ARCHER, CSD3, BNL}
\cvdoubleitem{Software}{\textbf{vim}, LaTeX, Inkscape, GIMP}{Utilities}{\textbf{git}, Docker, GNU toolchain}
\cvdoubleitem{Data viz}{\textbf{matplotlib}, Bokeh, d3.js, seaborn}{Web}{\textit{JavaScript, HTML, CSS}}

\section{\texttt{(CONFERENCES + PRESENTATIONS)}}
\cvitem{2018}{Total Energy and Force Methods, Poster Presentation, University of Cambridge}
\cvitem{2017}{\emph{Crystal structure prediction for next-generation battery anodes} (\colourhref{http://www.tcm.phy.cam.ac.uk/~me388/ss_11.17/}{slides}), Invited Talk, Solid State Seminar Series, University of Cambridge}
\cvitem{}{Second conference of Research Software Engineers, Volunteer, University of Manchester}
\cvitem{}{CASTEP Developer Workshop, Demonstrator and Poster Presentation, University of Oxford}
\cvitem{}{13th RSC Conference in Materials Chemistry, Poster Presentation (\colourhref{http://www.tcm.phy.cam.ac.uk/~me388/posters/mc13.pdf}{link}), University of Liverpool}
\cvitem{}{STFC Annual Battery Meeting, Attendee, Abingdon}
\cvitem{}{CCP9 Young Researchers Event, Poster Presentation, University of Cambridge}
\cvitem{}{Scientific Computing Day, Poster Presentation, University of Cambridge}
\cvitem{2016}{High Performance Computing Autumn Academy, Presenter, University of Cambridge}
\cvitem{}{SMARTER5, Poster Presentation, University of Bayreuth, Germany}
\cvitem{}{CASTEP Workshop, Demonstrator and Poster Presentation, University of Oxford}
\cvitem{}{CCP9 Young Researchers Event, Poster Presentation, University of York}
\cvitem{2015}{High Performance Computing Autumn Academy, Attendee, University of Cambridge}
\cvitem{}{CASTEP Workshop, Attendee, University of Oxford}

\pagestyle{fancy}
\nocite{*}
\printbibliography[title={\texttt{PUBLICATIONS}}]
\section{\texttt{REFEREES}}
\cvitem{}{Dr Andrew Morris, University of Birmingham; \colourhref{mailto:ajm255@cam.ac.uk}{\texttt{ajm255@cam.ac.uk}}}
\cvitem{}{Prof Francisco Guinea, University of Manchester; \colourhref{mailto:paco.guinea@icm.csic.es}{\texttt{paco.guinea@icm.csic.es}}}
\cvitem{}{Dr Paul Walmsley, University of Manchester; \colourhref{mailto:paul.walmsley@manchester.ac.uk}{\texttt{paul.walmsley@manchester.ac.uk}}}

%\clearpage
%\pagestyle{empty}
%\input{../summer_2k18_internships/faradion/faradion_cover_letter.tex}
\end{document}
